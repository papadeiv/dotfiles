\documentclass[./main.tex]{subfiles}
\begin{document}

% Slide # 2
\begin{frame}[label=slide02]
        % Title
        \frametitle{Second slide}

        % Body
        \footnotesize
        This is an example text to show how the body of the slide is rendered on the final pdf.
        \begin{definition}[Example definition]
                A rational number $n$ is any number that can be written as the fraction of two integres.
        \end{definition} \pause
        \begin{theorem}[Example theorem]
                This is definitely true \footnotemark[1]
        \begin{equation*}
                \sum_k\mathcal{S}_{k_x}(z) \approx \frac{S(z)^x}{k / 23 -\zeta\gamma [45- S(z)] + \ln(y) - j^2+x(l)}
        \end{equation*}
        \end{theorem} \pause
        \begin{example}[Example example]
                I dunno man \footnotemark[2].
        \end{example}

        \footnotetext[1]{\vspace{0cm}\tiny Kramers, H., Physica, 7 (1940)}
        \footnotetext[2]{\vspace{-0.5cm}\tiny Kuehn, C., Physica D, 240 (2011)}
\end{frame}

\end{document}
